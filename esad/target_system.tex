\section{Target System}

The experiment will use both solid ($^{12}$C) and cryogenic (LH$_2$) targets. The solid 
target, a 0.5-mm-thick carbon wire, will be mounted on the fork of the 2H01 wire harp. The 
wire harp will be mounted at the nominal location of the CLAS12 target. The fork with wires 
(30~mm spacing between arms of the fork) is mounted on a shaft connected to a stepper 
motor motion system. The system is aligned in a way that the beam is centered between the 
fork arms, which allows the target to be moved in and out of the beam without interference 
of the support frames.
   
The cryogenic target system is a joint product of Saclay (France) and JLab (the same target 
system that has worked for 15~years with the CLAS detector). The target cell is a $\sim 20$~mm 
diameter, 5-cm-long Kapton tube with $30~\mu$m entrance and exit windows. There is a 
heat-shield made of a 65~$\mu$m-thick-aluminized Mylar around the target cell. The cell is 
connected to the condensing system with three $\sim$1.5-m-long SS pipes. The whole system is 
in the vacuum. The target cell is inside a Rohacell scattering chamber, with a wall thickness 
of 6.5~mm, that is connected to the upstream beampipe. The scattering chamber has a 45~cm long,
25~mm diameter aluminum extension pipe, with a wall thickness of 1~mm. The thickness of the 
aluminum exit window on the extension tube is 50~$\mu$m.
               
\subsection{Hazards} 

The LH$_2$ target contains a condensed cryogenic fluid and is considered a pressure vessel. 
Sudden warming of the target due to a vacuum breach could result in rapid expansion of the 
target fluid. The system is designed to safely vent the excess pressure unless the vent 
lines are blocked by frozen hydrogen and/or frozen contaminates in the gas. Failure of the 
foam, aluminum extension, or the thin window of the scattering chamber could produce a loud 
noise and could result in a failure of the target integrity. The target utilizes flammable 
gas (hydrogen) during operation. Failure of the system could release flammable gas into the 
hall. The hydrogen target gas and the helium used in the target refrigerator are potential 
ODH risks, and failure of either system could reduce the oxygen levels in the hall.

There are no hazards associated with moving the solid target into the beam. The stepping 
motor linear actuator will be operated using EPICS controls. The GUI for operation of the 
target will have preset coordinates. 

\subsection{Mitigations}

The cryotarget system has been used for about 15~years with the CLAS detector in Hall~B. The 
design and construction of the new target cell and the scattering chamber are in accordance 
to AMSE standards. During operation the foam scattering chamber, the extension, and the thin 
window are surrounded by the Hall~B CLAS12 Central Detectors and are therefore difficult to 
access. A protective shield will be placed around the scattering chamber whenever the target 
is retracted from the Central Detector system and is under vacuum. Personnel working near the 
target shall wear hearing and eye protection whenever the foam extension and window are 
exposed and the system is under vacuum. No cold cryogenic components are accessible by personnel. 

Two H$_2$/D$_2$ gas detectors are installed near the target location, one above the right hand 
side of the electronics rack and another above the cryostat. In case of a detected leak, the 
control system will immediately warm up the whole target system and empty the target cell.

Relief valves installed in parallel with all of the remote control valves ensure that the 
safety system is entirely passive. In case of an increased pressure in the vacuum chamber, 
the cryotarget controller will stop all vacuum pumps on the cryotarget, close the 
FV\_D2T(FV\_H2T) and all $^4$He valves, and sound an alarm (located on top of the terminal 
in the Hall~B Counting House).

The quantity of flammable gas (H$_2$) is less than 80~g and is therefore considered a class-O 
installation ($<$600~g) and the rules and regulations for this installation shall be followed, 
notably:

\begin{itemize}

\item The area shall be posted ``Danger Flammable Gases.  No Ignition Sources",

\item Combustibles and ignition sources shall be minimized within 10~ft or 3~m of target's gas 
handling equipment and piping.
\end{itemize}

The target does not operate in a confined space, and the total quantity of hydrogen/helium in 
the system is under 1000 standard liters. This presents a negligible oxygen deficiency risk in 
Hall~B and therefore is a class-0 ODH installation.

Hydrogen shall be loaded into the system by qualified personnel only, and those personnel shall 
follow approved operational gas handling procedures.   Upon loss of target gas pressure, the 
control system shall automatically warm up the target to keep the pressure above atmospheric to 
prevent contamination. 

The target control software includes numerous alarms (temperature, pressure, vacuum, heater 
power, etc.) to alert users to potential problems. The solid target ladder incorporates limit 
switches to prevent motion outside of the allowed range if the set limits in EPICS fail. 

\subsection{Responsible Personnel}

The target system will be maintained by the Hall~B Engineering Group.  

\begin{table}[!htb]
\centering
\begin{tabular}{|c|c|c|c|c|}
\hline
 Name&Dept.&Phone&email&Comments \\ \hline
Engineering on call & Hall~B&(757)-748-5048&& 1st contact  \\ \hline
K. Bruhwel& Hall~B&x7868&\href{mailto:bruhwel@jlab.org}{\nolinkurl{bruhwel@jlab.org}}&2nd contact \\ \hline
D. Tilles & Hall~B&x7566&\href{mailto:tilles@jlab.org}{\nolinkurl{tilles@jlab.org}}  &3rd contact \\ \hline
R. Miller &Hall~B&x7867&\href{mailto:rmiller@jlab.org}{\nolinkurl{rmiller@jlab.org}} &4th contact \\ \hline
\end{tabular}
\caption{Personnel responsible for the CLAS12 target system.} 
\label{tb:target}
\end{table}
