\section{Silicon Tracker}

The CLAS12 SVT is a barrel-shaped tracking detector that has a wide azimuthal angular 
coverage and $\sim 2\pi$ coverage in polar angle. It has four polygonal regions, R1 - R4, 
that have 10, 14, 18, and 24 sectors, respectively. Each sector contains modules, whose top 
and bottom sides have three, 320-$\mu$m-thick, silicon sensors that are wire bonded together, 
a pitch adapter, and a readout hybrid - part of the readout electronics located on the hybrid 
flex circuit board (HFCB). The bottom side of the module, closer to the beam, is referred to 
as the U layer; the top side of the module is referred to as the V layer. Each side of the 
module has 256 readout strips. 

Module services provide power, cooling, and communication to all the SVT modules. The services 
connected to the SVT include: power supply cables, data and control cables, cooling pipes, and 
cables for monitoring humidity and temperature. The power supply system consists of low voltage 
and high voltage supplies, and MPOD crates. The low voltage supply powers the analog and digital 
portions of the readout chips. The high voltage supply is for biasing the sensors and monitors 
the leakage current over a wide range. The normal operating point of reverse bias for the SVT 
sensor is 85~V. Each side of the module receives low voltage, 2.5~V for both the analog and 
digital parts of the FSSR2 chip and high voltage for the sensors. The low voltage also powers 
analog output CMOS IC temperature sensors, one per side.

The front-end chips of the SVT modules have to be cooled to ensure normal operating conditions. 
The cooling system of the SVT consists of the portable chiller, plastic cooling tubes, flow
meters, and the cold plates with copper tubes inside circulating liquid coolant. The SVT 
dry air  purging system is designed to provide a dry environment inside the detector and to 
avoid condensation.

\subsection{Hazards} 

Hazards to personnel include the high voltage that biases the sensors and the low voltage 
current that powers the readout electronics. During the installation phase, mechanical hazards 
include the risks associated with lifting the SVT during integration, transportation, and 
installation, and the work at height in order to access the crates and patch panel on the 
insertion cart. 

Hazards to the SVT itself include mechanical damage, radiation damage, gas over-pressure, 
cooling system leaks, and overheating. Overheating can occur in the SVT  if the cooling 
system is performing inadequately or if a cooling system leak develops.

Radiation damage could occur in the SVT if the beam moves into the sensors, the beam interacts 
upstream to produce excessive radiation, or excessive beam currents create more radiation than 
can be tolerated. 

Wrong LV settings could damage the hybrids and ambient sensors; wrong HV settings could cause 
high leakage current that could damage the sensors.

Failure of the crate cooling fans could cause overheating and damage of the crate modules.

Overpressure in the cooling lines could cause damage of the cooling system.

\subsection{Mitigations}

Electrical hazards (personnel):
\begin{itemize}
\item Hazards to personnel are mitigated by turning off HV and LV power before disconnecting 
cables or working on the sensors and electronics. 
\end{itemize}

Mechanical hazards (personnel):
\begin{itemize}
\item Hazards related to work at height during installation and maintenance is mitigated by 
proper safety training and using certified step ladders or scaffolding. 
\item Mitigation of hazards to the personnel related to the SVT lifting is done by admitting 
only trained JLab staff, using personal protection and certified gantry, cranes, and tooling.
\end{itemize}

Mechanical hazards (detector):
\begin{itemize}
\item Possible mechanical damage has been mitigated by following the proper procedures for each 
job and using the special tooling. 
\item All operations related to handling the SVT are performed only by trained personnel with 
hands on experience working with the SVT. 
\item Lifting the SVT during the installation and maintenance is performed only by properly 
trained personnel using certified equipment. 
\item SVT integration with the MVT is done only by the trained personnel following the proper 
procedures. 
\end{itemize}

Radiation damage hazards (detector):\\

Radiation damage from the beam is mitigated in several steps. 
\begin{itemize}
\item Beam size and halo must conform to beam requirements before beam is passed through the 
detector. 
\item An upstream collimator is aligned with the ``centered" beam position to intercept the 
beam if it moves off nominal position. 
\item Beam halo monitors sense a rise in backgrounds if the beam moves off its nominal position, 
activating the beam Fast Shutdown (FSD). 
\end{itemize}

Other detector hazards:
\begin{itemize}
\item Cooling system: Overheating is mitigated by leak checks, requiring good coolant flow and 
pressure, proper coolant temperature, and sensor temperatures in the working range. The cooling 
system is constantly monitored by an EPICS IOC, logged to a MYA database, and interlocked to 
the HV and LV power systems, and the alarm handler. There are interlocks on coolant leaks, 
hybrid temperature and humidity, ambient temperature, humidity, and dew point. All temperature 
and humidity sensors are redundant.
\item Electronics and sensors: Power supplies have hardware and software limits set, and currents 
and voltages are controlled, monitored, and included in the alarm handler. Crate temperatures 
are monitored by the control system. The proper HV/LV power supply ramp up and ramp down sequence 
is ensured by the Slow Controls software to prevent human mistakes. 
\item Gas system: To prevent condensation and gas over-pressure, the gas flow of the nitrogen 
purging is controlled, interlocked, and monitored by the Slow Controls system. 
\item The hardware interlock system provides redundant safety for critical parameters in case of 
alarm handler failure. 
\item Control system parameters and settings can be saved and restored.
\end{itemize}

\vfil
\eject

\subsection{Responsible Personnel}

Individuals responsible for the SVT system are:

\begin{table}[htbp]
\centering
\begin{tabular}{|c|c|c|c|c|} \hline
Name&Dept.&Phone&email&Comments \\ \hline
Expert on call& &&& 1st contact \\ \hline
Y. Gotra& JLab&757-269-5571&\href{mailto:gotra@jlab.org}{\nolinkurl{gotra@jlab.org}}&2nd contact \\ \hline
B. Eng&JLab&757-269-6018&\href{mailto:beng@jlab.org}{\nolinkurl{beng@jlab.org}}&3rd contact \\ \hline
R. Paremuzyan&JLab&757-541-7539&\href{mailto:rafopar@jlab.org}{\nolinkurl{rafopar@jlab.org}}&4th contact \\ \hline
\end{tabular}
\caption{Personnel responsible for the CLAS12 SVT system.} 
\label{tb:svt}
\end{table}

