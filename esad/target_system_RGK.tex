\section{Target System}

The target system used for RG-K is the Hall B cryotarget. RG-K will only use liquid-hydrogen in the Hall B
cryotarget system that employs a 5-cm-long cell. A scattering chamber is installed around the target cell
area made from Rohacell foam with a wall thickness of 6.5~mm. Aluminum windows are used at the entrance and
exit of the liquid cell, and at the exit of the scattering chamber. The details of all components, such as
windows and cells, are shown on the beamline drawing, including thicknesses and locations. These drawings can
be found at Ref.~\cite{engineering-page}.

\subsection{Hazards} 

The cryogenic target contains a condensed cryogenic fluid and is considered a pressure vessel. Sudden warming
of the target due to a vacuum breach could result in rapid expansion of the target fluid. The system is designed
to safely vent the excess pressure. Failure of the foam scattering chamber, or the thin window of the scattering
chamber could produce a loud noise and could result in a failure of the target integrity.

The target utilizes flammable gas (hydrogen) during operation. Failure of the system could release flammable gas
into the hall. The target gases and the helium used in the target refrigerator are potential ODH risks, and failure
of either system could reduce the oxygen levels in the hall.

The target cell system is protected by 15 psi relief valves. The target operates in a vacuum chamber, so the total
pressure difference possible across the cell is 15~psi~+~14.7 ~psi=~29.7~psi. The cell is considered a pressure
vessel. If the Kapton cell ruptures, the target gas would vent into the target vacuum space and the vacuum pumps
would turn off. If the vacuum space pressure increases to 1 psi, the target gas will go out of the vacuum space
relief valve and be discharged out of the Hall. No target gas would enter the Hall.

\subsection{Mitigations}

The design and construction of the entire Hall B cryotarget is in accordance with AMSE standards. During operation,
the foam scattering chamber, and the thin window are surrounded by the Hall-B CLAS12 Central Detectors and are
therefore difficult to access.~A protective shield will be placed around the scattering chamber whenever the target
is retracted from the Central Detector system and is under vacuum.  Personnel working near the target shall wear
hearing and eye protection whenever the foam extension and window are exposed and the system is under vacuum. No
cold cryogenic components are accessible by personnel.

Relief valves are installed in all the target pressure circuits so the safety system is entirely passive.~The quantity
of flammable gas (H2) is less than 80~g and is therefore considered a class-0 installation ($<$600~g) and the rules and
regulations for this installation shall be followed, notably:

\begin{itemize}
\item The area shall be posted ``Danger Flammable Gases. No Ignition Sources" 
\item Combustibles and ignition sources shall be minimized within 10 ft or 3~m of target’s gas handling equipment and
  piping.
\end{itemize}

The target does not operate in a confined space, and the total quantity of hydrogen/helium in the system is under 1000
standard liters. This presents a negligible oxygen deficiency risk in Hall B and therefore is a class-0 ODH installation.
Hydrogen shall be loaded into the system by qualified personnel only, and those personnel shall follow approved operational
gas handling procedures.

The target control software includes numerous alarms (temperature, pressure, vacuum, heater power, etc.) to alert users
to potential problems.

\subsection{Responsible Personnel}

The target system will be maintained by the JLab Target Group.  

\begin{table}[!htb]
\centering
\begin{tabular}{|c|c|c|c|c|}
\hline
Name & Department & Phone & email & Comments \\ \hline
Target on call & Hall B & 757-218-2266 & & 1st contact \\ \hline
Xiangdong Wei  & Hall B & 516-635-1957 & \href{mailto:xwei@jlab.org}{\nolinkurl{xwei@jlab.org}} & 2nd contact \\ \hline
Engineering on call & Hall~B&757-748-5048&& 3rd contact  \\ \hline
\end{tabular}
\caption{Personnel responsible for the CLAS12 target system.} 
\label{tb:target}
\end{table}
