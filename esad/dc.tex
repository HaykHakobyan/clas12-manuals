\section{Drift Chambers}

The CLAS12 Drift Chamber (DC) system is comprised of 18 separate chambers. There are 
three types: ``region 1", ``region 2", and ``region 3" depending on location upstream, 
within, or downstream of the CLAS Torus magnet. Each chamber has wires arranged in two 
superlayers of 6~layers by 112~wires. The gas system supplies mixed, clean, 
pressure-controlled argon/CO$_2$ gas to each of the 18 drift chambers. The on-chamber 
amplifier and readout boards are called ``signal translator boards" (STBs). There are 7 such 
boards per superlayer. They distribute low voltage (LV) power to pre-amplifiers located on the 
board, one for each sense wire. The pre-amps are placed in groups of 16, with six such groups 
per board. There is an individual fuse for every group of sixteen. Thirty-four conductor signal 
cables (16 twisted-pair signals) connect each STB group of 16 pre-amps with one connector on 
the drift chamber readout board (DCRB). High voltage (HV) is supplied to the wires by on-chamber 
``high-voltage translator boards'' (HVTBs), located on the opposite endplate from the STBs. The 
high voltage is supplied to the HVTBs by a chain of cables connecting the HV crates to the 
``high voltage distribution boards'' (HVDBs) and from there by cables to the HVTBs.

\subsection{Hazards} 

Hazards to personnel include the high voltage supplied to the wires and the low voltage that 
powers the on-chamber pre-amplifiers. Hazards to the drift chambers themselves include damage 
to the gas windows should the pressure deviate more than a few psi from atmospheric. 

\subsection{Mitigations}

Electrical hazards:
\begin{itemize}
\item High Voltage: high voltages up to 2000~V are used routinely for all detectors. Mitigation: 
very low current limits (40~$\mu$A) are set. All mechanical structures are properly grounded.  
There are possible electrical hazards if a malfunctioning HV board is replaced. The associated 
task hazard analysis concluded that the consequence level is low, the probability level is low, 
and the risk code is 1. The mitigation for these electrical jobs is to place the equipment in 
Mode~0 (de-energized) when replacing or repairing hardware during routine maintenance. The 
same risk codes and mitigation applies to the procedure known as a ``minimal disconnect'' in 
which an individual HV pin is removed from the HV distribution box.

\item Low Voltage: In order to power up the on-chamber electronics, we use low voltage at 7~V 
with 50~A per supply (1 supply per chamber). Mitigation: voltage is low enough not to be a 
danger to personnel. All mechanical structures are properly grounded. All cables and connectors 
are certified for this rating and shielded. To protect against possible over-heating of the 
on-chamber pre-amplifier boards, each individual conductor (positive and neutral return) is 
fused; with the fuses located in a fuse panel with a red LED signaling a blown fuse. If a 
fuse is removed and/or replaced there is no risk to personnel because of the low voltage.

\end{itemize}

Gas system hazards:
\begin{itemize}
\item Personnel: because most of the system operates very close to atmospheric pressure there 
is no hazard to personnel in the hall due to pressure. The gas is non-toxic and non-flammable.  
Because of the large volume of the hall and the location of the chambers in the main open area 
of the hall, there is no ODH hazard to personnel.
\item Detectors: there is a potential danger to the chamber gas windows if the pressure in the 
chamber differs from atmospheric by one psi. This is mitigated during standard operation by our 
pressure-difference control system with fail-safe over-pressure and under-pressure bubblers 
providing an additional level of safety.
\end{itemize}

\subsection{Responsible Personnel}

Individuals responsible for the DC system are:

\begin{table}[!htb]
\centering
\begin{tabular}{|c|c|c|c|c|} \hline
Name &Dept.&Phone&email&Comments \\ \hline
Expert on call &       &       &       & 1st contact \\ \hline
Mac Mestayer&Hall~B&(757)-229-6575& \href{mailto:mestayer@jlab.org}{\nolinkurl{mestayer@jlab.org}} &  2nd contact\\ \hline
Morgan Cook&Hall~B&  & \href{mailto:mcookiv@jlab.org}{\nolinkurl{mcookiv@jlab.org}} & 3rd contact\\ \hline

 \end{tabular}
\caption{Personnel responsible for the CLAS12 DC system.} 
\label{tb:dc}
\end{table}

Cook, Morgan	TED 1546-19	 	mcookiv

