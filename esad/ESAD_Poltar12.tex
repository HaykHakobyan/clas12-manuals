
\section{ Longitudinally Polarized Target}

The CLAS12 longitudinally polarized target will be utilized for Run Group C.
This consists of small (5 g) samples of frozen ammonia (NH$_3$) or
deuterated-ammonia (ND$_3$) cooled to 1 K by a $^4$He evaporation refrigerator,
exposed to a 5~T magnetic field, and polarized by spin flip
transitions induced by 140~GHz microwaves.  Liquid helium for the refrigerator is supplied
from the Hall B buffer dewar, while the 5~T magnetic field
is generated by the CLAS12 solenoid.  The hazards associated with these two
devices and their mitigations are discussed in other sections of this ESAD.

Seven potential hazards and their mitigations are considered here: ammonia handling, cryogenic, electrical, vacuum, pressure, oxygen deficiency, and RF radiation.  

\subsection{Ammonia handling hazard and mitigation} 
{\bf Hazard:} In its gaseous form ammonia is classified as a hazardous chemical with an OSHA 8-hour exposure limit of 25~ppm by volume in air.  

{\bf Mitigation:} The target material is maintained in its frozen state (melting point -78~C) at all times using either liquid nitrogen or liquid helium.  The target quantity is small, approximately 5 g, and upon evaporation will expand to a gaseous volume of about 0.006~m$^3$.  Ammonia gas is lighter than air and will be dispersed throughout the volume of Hall B (12,800 m$^3$).  The resulting concentration will be 0.5~ppm, far below the OSHA limit of 25~ppm.  The gas has a highly pungent odor that serves as a warning of potentially dangerous exposure. The average odor threshold is 5 ppm, again well below the OSHA limit.

\subsection{Cryogenic hazard and mitigation}
 {\bf Hazard:} Liquid nitrogen is used to store and maintain the sample in its frozen state.  Exposure to liquid nitrogen can result in severe cryogenic burns.  
  
{\bf Mitigation:} Appropriate personal protection equipment (PPE) is worn whenever the containers are removed from LN2 storage and loaded into the target cryostat.   The PPE consist of glasses, face mask, long sleeves and pants, closed-toed shoes, and cryogenic apron and gloves. The sample is removed from the storage dewar and placed in the polarized target refrigerator using shop-made tools specifically designed to eliminate contact with the cold sample or the LN2.  


\subsection{Electrical hazard and mitigations}
{\bf Hazard:} The target instrumentation consists of numerous devices powered by 120 VAC.  The power supply for the microwave source is 208 VAC, and the roots pumping system for the evaporation refrigerator is 480 VAC.  The microwave source develops voltages up to 10 kV.  Hazards commensurate with these voltages include lethal shocks and burns as well as the potential to start fires in nearby combustible materials.

{\bf Mitigation:} All electrical equipment will be installed and maintained by a qualified JLab electrical worker.  The microwave source is protected from personnel exposure by an interlocked enclosure that trips off the
power supply when opened.  A wiring junction box between the source and power supply is similarly interlocked.
 
\subsection{Vacuum hazard and mitigation}
{\bf Hazard:} The target refrigerator is surrounded by a chamber made of stainless steel, aluminum, and
carbon fiber and is evacuated to less than 10$^{-5}$~torr during normal operation.   A single aluminum
window (2-mil thick, 2.5~ cm diameter) for beam exit is located at the downstream of the chamber.  
Loud noise caused by rupture of this window may result in hearing loss.

{\bf Mitigation:} Under normal operation, this window is surrounded by the CLAS12 detector systems and is not accessible by personnel.  A protective cover shall be in place if the chamber is evacuated and the window exposed to personnel.  Hearing protection will be required for all personnel working in the vicinity of the chamber when the chamber is evacuated and the thin exit window is exposed.


\subsection{Pressure vessel hazard and mitigation}
{\bf Hazard:} The target refrigerator contains approximately 1~liter of liquid helium that could present an
overpressure and explosion hazard if the surrounding vacuum chamber is breached.  Room temperature
helium gas from the ESR 4~atm helium supply is used to purify the refrigerator piping systems prior
to cooling.  This gas could overpressurize and rupture refrigerator components.

{\bf Mitigation:} Upon warming, the liquid helium will expand to a volume of approximately 1~m$^3$ but will be contained within the larger refrigerator volume at a pressure under one atmosphere.  Nevertheless the refrigerator and all its piping systems are designed and constructed in accordance with the JLab pressure-vessel safety standards.  Appropriately-sized, ASME-approved and non-defeatable relief valves are installed on all sections of target piping to eliminate over-pressurization.  

\subsection{Oxygen deficiency hazard and mitigation}
 {\bf Hazard:} Normal operation of the target relies on two cryogenic fluids, liquid helium and liquid nitrogen,
 which expand and displace oxygen upon warming.  The resulting decrease of O$_2$ levels can be harmful
 to personnel in the area. 
 
 {\bf Mitigation:}  All helium gas processed through the polarized target system is returned to ESR for liquefaction or piped outside via the Hall B vent header.  The volume of liquid nitrogen used to store ammonia target sample is 35~liters.  Upon warming, the liquid will expand to 25~m$^3$ inside the 13,000 m$^3$ volume of Hall B, resulting in less than 0.2\% reduction in oxygen levels.  .  The volume of liquid helium in the refrigerator is about 1~liter, which expands to less than 1~m$^3$.  This reduces the O$_2$ percentage by less than 0.01\%.  Neither are therefore an ODH risk.  

\subsection{RF hazard and mitigation}
{\bf Hazard:} The target samples are polarized using a 140~GHz microwave source with a maximum output
of about 20~W.  Exposure to this radiation could result in severe burns as a result of local heating.  

{\bf Mitigation:}  The microwaves are contained in metal waveguides or cavities at all points, and
direct exposure to the skin or eyes is not possible.  Portions of waveguide that may be warm to the touch
are covered by the same interlocked, protective enclosure as the microwave source.

\subsection{Responsible Personnel}
Individuals responsible for the polarized target are:
\begin{table}[h]
\begin{tabular}{|c|c|c|c|c|}
\hline
Name           & Dept.  & Phone        & email           & Comments    \\ \hline
Expert on call &        & tbd          &                 & 1st contact \\ \hline
Chris Keith    & Target & 757-746-9277 & ckeith@jlab.org & 2nd contact \\ \hline
\end{tabular}
\end{table}
%\end{document}